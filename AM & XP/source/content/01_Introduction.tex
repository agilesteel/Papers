\section{Introduction}
The software industry is a very young craft. Software is a very
fragile artifact. The software process is unstable and software methodologies
are controversial. Software projects contain many defects, do not accomplish
the tasks they were meant to accomplish and are delivered late, when delivered at
all. Every stakeholder of a software project is unhappy, loses money and this
clearly has to change.

Over the years a lot of effort has been made to address these problems. Scrum,
XP, Dynamic Systems Development Method (DSDM), or Feature-Driven Development
(FDD) and many others are steps in the right direction. These methodologies vary
in many aspects, but agree on the fundamental level with each other. On
February 11-13, 2001 seventeen software professionals, among them Jeff
Sutherland - the inventor of Scrum and Kent Beck - the inventor of XP met to
find some common ground, which would lead the software industry into the future.
Naming themselves \enquote{The Agile Alliance}, these people handcrafted a
document, which is by now known as \enquote{The Manifesto for Agile Software
Development} or simply \enquote{The Agile Manifesto}. Declaring a list of
principles, this document has become a beacon of hope for the software industry.
Years passed. The term \enquote{agile} is polluted by the smog of certification
madness, overwhelming quantity of low quality information sources, religious
ignorance and simple lack of understanding the simple principles those seventeen
individuals put together over 10 years ago.

This paper revisits XP and augments its modeling part by explaining AM, which
is used to reinforce the values of agile software development and clear the myth
of extreme programmers being old fashioned, spoiled hackers, who just do not
know better.

