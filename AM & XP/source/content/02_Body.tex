\section{Scrum is not just about standing up and burning down}
This section highlights the most important principles of the Agile
Manifesto.

{\bf Individuals and interactions over processes and tools.} Developing
software is an intellectual and creative process, which is challenging and spectacularly
complicated. Managers do not make any differentiation between a software
developer and a house builder. They assume they can plug and play people into a
software project as they can in the house building industry, which is obviously
not the same thing. Agile teams are most commonly compared to
sports teams, which can only win if the team members play together. Teamwork
requires by definition a great deal of communication skills. Processes are
important, but not vital. If you had a choice between an NBA level basketball
team and a group of very smart, athletic and toll people who never heard of
basketball before, but got it explained in detail just prior to the match, on
which team would you bet your money on? And always remember, that although tools
are helpful a fool with a tool is still a fool.

{\bf Working software over comprehensive documentation.} The goal of
software development is software, not documentation, otherwise it would be called
documentation development.\cite{Ambler200204} Documentation is important, but
not a priority. And why should it be? If you think about it, working software is
even more explanatory than a technical, gibberish document.

{\bf Customer collaboration over contract negotiation.} Developing software is a
creative process and the fact that software developers are the only stakeholders
in the software project is nothing more than an illusion. The customer {\bf
develops} ideas and the software developer implements them. Thus the customer
becomes a vital part of the team and communication skills are required yet
again. The customer has to be creative himself. Since creativity is an iterative
process, there is just no way, the customer could communicate his wishes to the
software developer upfront. And even if he could, would the software developer
go ahead and build the whole system in one sit?

{\bf Responding to change over following a plan.} The fact that change {\bf is}
the state in the software world is undeniable. Technology, business environment,
skills, domain understanding change over time and there is nothing you can do
about it. In fact you should embrace change and act now instead of planning for
tomorrow to paraphrase the surprisingly effective way of thinking:
\enquote{Implement for today, design for tomorrow.}\cite{BeckAndres200411}

\section{Extreme Programming}