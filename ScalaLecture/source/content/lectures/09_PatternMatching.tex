\part[Pattern Matching]{Pattern Matching}
\section{Objects VS Data Structures}
\begin{frame}[fragile]{Objects VS Data Structures}
\begin{block}{Objects}
\begin{lstlisting}
abstract class Shape {
   def draw(): Unit
}
class Circle extends Shape {
   def draw(): Unit = println("Drawing a circle")
}
class Rectangle extends Shape {
   def draw(): Unit = println("Drawing a rectangle")
}   
\end{lstlisting}
\end{block}
\pause
\begin{block}{Algorithm which uses objects}
\begin{lstlisting}
val shape: Shape = getShape
shape.draw()
\end{lstlisting}
\end{block}
\end{frame}

\begin{frame}[fragile]{Objects VS Data Structures}
\begin{center}
Adding new types does not require the algorithm to be modified, recompiled,
retested or redeployed. Adding new methods to \lstinline!Shape! on the other
hand requires modification, recompilation, retesting and redeployment of
\alert{all} \lstinline!Shape! derivatives.
\end{center}
\end{frame}

\begin{frame}[fragile]{Objects VS Data Structures}
\begin{block}{Data Structures}
\begin{lstlisting}
class Circle
class Rectangle
object Shape {
   def drawCircle(circle: Circle): Unit =
      println("Drawing a circle")
   def drawRectangle(rectangle: Rectangle): Unit =
      println("Drawing a rectangle")
}
\end{lstlisting}
\end{block}
\end{frame}

\begin{frame}[fragile]{Objects VS Data Structures}
\begin{block}{Algorithm which uses data structures}
\begin{lstlisting}
val shape: Any = getShape
if (shape.isInstanceOf[Circle])
   Shape.drawCircle(shape.asInstanceOf[Circle])
else if (shape.isInstanceOf[Rectangle])
   Shape.drawRectangle(shape.asInstanceOf[Rectangle])
else 
   println("Not a shape")
\end{lstlisting}
\end{block}
\end{frame}

\begin{frame}[fragile]{Objects VS Data Structures}
\begin{center}
Adding new methods does not require the algorithm to be modified, recompiled,
retested or redeployed. Adding new types of shape on the other
hand requires modification, recompilation, retesting and redeployment of
\alert{all} algorithms, that use shapes.
\end{center}
\end{frame}

\begin{frame}{Objects VS Data Structures}
\onslide<1->\begin{block}{Objects}
\onslide<2->Objects have \highlight{state} (data) and \highlight{behavior} (methods)\\
\onslide<3->Objects \alert{hide state} and \highlight{expose behavior}\\
\onslide<4->Objects protect you against \highlight{new types}\\
\end{block}
\begin{center}
\onslide<5->Objects and Data Structures are \highlight{diametrically opposed}
\end{center}
\onslide<1->\begin{block}{Data Structures}
\onslide<2->Data Structures have \highlight{state} (data) but \alert{no behavior}* (methods)\\
\onslide<3->Data Structures \highlight{expose data} and \alert{``hide'' behavior}\\
\onslide<4->Data Structures protect you against \highlight{new behavior}\\
\end{block}
\onslide<2->* Data Structures may contain \highlight{navigational} methods
\end{frame}

\begin{frame}[fragile]{Alternatives for dealing with Data Structures}
\begin{block}{Method Overloading}
\begin{lstlisting}
class Circle
class Rectangle
object Shape {
   def draw(circle: Circle): Unit =
      println("Drawing a circle")
   def draw(rectangle: Rectangle): Unit =
      println("Drawing a rectangle")
}
\end{lstlisting}
\end{block}
\pause
\begin{alertblock}{Algorithm which uses the data structure}
\begin{lstlisting}
val shape: Circle = getShape
Shape.draw(shape) // the type needs to be known at compile time
\end{lstlisting}
\end{alertblock}
\end{frame}

\begin{frame}[fragile]{Alternatives for dealing with Data Structures}
\begin{block}{Match expression}
\begin{lstlisting}
val shape: Any = getShape
shape match {
   case circle: Circle => Shape.draw(circle)
   case rectangle: Rectangle => Shape.draw(rectangle)
   case somethingElse => println("Not a shape")
}
\end{lstlisting}
\end{block}
\end{frame}

\begin{frame}[fragile]{Match expressions are switch statements on steroids}
\begin{block}{Switch statement in Java}
\lstinline[language=java]!switch (x) { alternatives }!
\end{block}
\begin{block}{Match expression in Scala}
\lstinline!x match { alternatives }!
\end{block}
\begin{block}{Differences}
\begin{center}
\begin{tabular}{|l|r|}
\hline
\textsc{Java switch} & \textsc{Scala match}\\
\hline
\hline
statement & expression\\
\hline
requires breaks & does not fall through cases\\
\hline
works only for integers and enum types* & works for virtually everything\\
\hline
\end{tabular}
\end{center}
\end{block}
* strings are also supported since Java 7
\end{frame}
\section{Extractors}
\begin{frame}{Extractors}
Computers store data as ones and zeroes. Only literally hundreds abstraction
levels later we see emails, pictures, videos and so on. In other words,
we \highlight{view} the data representation but not the data itself. Programming
languages provide means for defining your own \highlight{data abstractions}. For
instance, a class can encapsulate few strings and integers. The end user will
view objects of this class, as, for example, an address, the programmer will
view this class as a bunch of strings and integers, the compiler will view this
class as a type. The list continues all the way down to ones and zeros.
\end{frame}

\begin{frame}{Extractors}
Each of those \highlight{viewers extract} some information from this class by
looking at it in a special way without knowing what they are actually looking
at. You can imagine it as looking  at something with different glasses. Each
pair of glasses is adjusted to \highlight{recognize a single pattern}. Scala
gives you tools for building such glasses called \highlight{extractors}.
\end{frame}

\begin{frame}[fragile]{Extractors}
We want to write an extractor called \lstinline!File!, which would parse file
paths, represented by \lstinline!String!s and extract the \highlight{location},
the \highlight{name} and the \highlight{extension} of a file path respectively.
The extractor should also take care of testing if the given \lstinline!String!
has a form of a file path.
\pause
\begin{exampleblock}{The \lstinline!File! extractor in action}
\begin{lstlisting}
selector match {
   case File(location, name, extension) => ...
}
\end{lstlisting}
\end{exampleblock}
\pause
Before that we take a quick look at \lstinline!Option! and \lstinline!Tuple!
from the Scala library\ldots
\end{frame}

\begin{frame}[fragile]{Tuple}
\begin{block}{What is a tuple?}
Tuples are heterogeneous collections with fixed size. Use them like classes
without names, but do not confuse them with anonymous classes!
\end{block}
\pause
\begin{exampleblock}{Tuples in action}
\lstinline!// Creating tuples!\\
\lstinline!val x = new Tuple2("hello", 5)!\\
\lstinline!val x = ("hello", 5)!\\
\lstinline!val x = "hello" -> 5!\\
\lstinline!val x: Tuple2[String, Int] = "hello" -> 5!\\
\lstinline!val x: (String, Int) = "hello" -> 5!\\
\lstinline!// Using tuples!\\
\lstinline!scala> val s = x._1!\\
\lstinline!s: String = hello!\\
\lstinline!scala> val i = x._2!\\
\lstinline!i: Int = 5!
\end{exampleblock}
\end{frame}

\begin{frame}[fragile]{Option}
\begin{block}{Option is a better way of handling \lstinline!null!}
\lstinline!sealed abstract class Option[+A]!\\
\lstinline!case class Some[+A](x: A) extends Option[A]!\\
\lstinline!case object None extends Option[Nothing]!
\end{block}
\pause
\begin{exampleblock}{Option in action}
\begin{lstlisting}
scala> val dogs = Array("Snoopy", "Fluffy")
dogs: Array[String] = Array(Snoopy, Fluffy)

scala> val maybeFluffy = dogs find { _ == "Fluffy" }
maybeFluffy: Option[String] = Some(Fluffy)

scala> val maybeRex = dogs find { _ == "Rex" }
maybeRex: Option[String] = None
\end{lstlisting}
\end{exampleblock}
\end{frame}

\begin{frame}[fragile]{The \lstinline!File! path extractor}
\onslide<1->\lstinline!object File {!\\
\onslide<7->\lstinline!   // The injection method (optional)!\\
\onslide<7->\lstinline!   def apply(loc: String, name: String, ext: String): String =!\\
\onslide<7->\lstinline!      loc + "\\" + name + "." + ext!\\
\onslide<7->\lstinline!!\\
\onslide<2->\lstinline!   // The extraction method (mandatory)!\\
\onslide<2->\lstinline!   def unapply!\onslide<3->\lstinline!(path: String)!\onslide<4->\lstinline!: Option[(String, String, String)]!\onslide<2->\lstinline! = {!\\
\onslide<5->\lstinline!      val location = // parse the path somehow!\\
\onslide<5->\lstinline!      val name = // parse the path somehow!\\
\onslide<5->\lstinline!      val extension = // parse the path somehow!\\
\onslide<6->\lstinline!      if (/* somethingWentWrong */) None!\\
\onslide<6->\lstinline!      else Some(location, name, extension)!\\
\onslide<2->\lstinline!   }!\\
\onslide<1->\lstinline!}!
\end{frame}

\begin{frame}[fragile]{The \lstinline!File! path extractor}
\begin{exampleblock}{File factory in action}
\begin{lstlisting}
scala> val implicitFile = """location\name.extension"""
implicitFile: String = location\name.extension

scala> val explicitFile = File("location", "name", "extension")
explicitFile: String = location\name.extension
\end{lstlisting}
\end{exampleblock}
\end{frame}

\begin{frame}[fragile]{The \lstinline!File! path extractor}
\begin{exampleblock}{File extractor in action 1}
\begin{lstlisting}
def isThisAFile(selector: Any): Boolean = selector match {
  case File(_, _, _) => true
  case _ => false
}
\end{lstlisting}
\end{exampleblock}
\pause
\begin{block}{File extractor in action 1 - some notes}
\begin{itemize}
  \item if \lstinline!selector! is not of type \lstinline!String! the first case does not match
  \item \lstinline!case File(_, _, _) =>! is transformed to \lstinline!File.unapply(selector)!
  \item \lstinline!case _ => ! is a default case (exception is thrown if not provided)
\end{itemize}
\end{block}
\end{frame}

\begin{frame}[fragile]{The \lstinline!File! path extractor}
\begin{exampleblock}{File extractor in action 2}
\begin{lstlisting}
def showNameIfFile(selector: String): Unit = selector match {
  case File(_, name, _) => println(name)
  case _ =>
}
\end{lstlisting}
\end{exampleblock}
\pause
\begin{block}{File extractor in action 2 - some notes}
\begin{itemize}
  \item no need to check if \lstinline!selector! is of type \lstinline!String!,
  but the syntax remains the same
  \item no need to return \lstinline!()! in the default case
\end{itemize}
\end{block}
\end{frame}

\begin{frame}[fragile]{The \lstinline!File! path extractor in action 3}
\onslide<1->\lstinline!def isTODOtxt_isOn_D_Docs(selector: Any): String = selector match {!\\
\onslide<2->\lstinline!  case File("""D:\Docs""", "todo", "txt") =>!\\
\onslide<2->\lstinline!    """Yes, todo.txt is on D:\Docs"""!\\
\onslide<3->\lstinline!  case File(location, "todo", "txt") =>!\\
\onslide<3->\lstinline!    "No, todo.txt is on " + location!\\
\onslide<4->\lstinline!  case File("""D:\Docs""", name, "txt") =>!\\
\onslide<4->\lstinline!    "No, but " + name + """ is on D:\Docs"""!\\
\onslide<5->\lstinline!  case File(location, name, "txt") =>!\\
\onslide<5->\lstinline!    "This is " + name + ".txt and it is on " + location!\\
\onslide<6->\lstinline!  case File(loc, name, ext) =>!\\
\onslide<6->\lstinline!    "This is " + name + "." + ext + " and it is on " + loc!\\ 
\onslide<7->\lstinline!  case file @ File(location, name, ext) => file.toString!\\
\onslide<8->\lstinline!  case x: String => "What am I supposed to do with " + x + "?"!\\
\onslide<9->\lstinline!  case _ => "This does not even look like a file"!\\
\onslide<1->\lstinline!}!\onslide<10->\lstinline!// The order of the alternatives is significant!
\end{frame}

\begin{frame}[fragile]{The \lstinline!File! path extractor}
\begin{exampleblock}{File extractor in action 4 - nested patterns}
\begin{lstlisting}
def deepMatch(selector: Any): Unit = selector match {
  case File(location @ Location("D:\\", _*), name, "txt") =>
     location foreach prinln
}
\end{lstlisting}
\end{exampleblock}
\pause
\begin{exampleblock}{File extractor in action 4 - nested patterns}
\begin{lstlisting}
val bunchOfPaths = List("loc1", "D:\docs\homework.txt", "loc3")
bunchOfPaths foreach deepMatch
\end{lstlisting}
\end{exampleblock}
\pause
\begin{block}{Note}
The \lstinline!Location! extractor needs to implement the\\
\lstinline!def unapplySeq(location: String): Option[Seq[String]]!
\end{block}
\end{frame}

\begin{frame}[fragile]{Kinds of patterns}
\onslide<1->\lstinline!selector match {!\\
\onslide<3->\lstinline!   case 5 => "constant pattern"!\\
\onslide<3->\lstinline!   case true => "constant pattern"!\\
\onslide<3->\lstinline!   case "hello" => "constant pattern"!\\
\onslide<4->\lstinline!   case Nil => "constant pattern"!\\
\onslide<5->\lstinline!   case x => "variable pattern"!\\
\onslide<5->\lstinline!   // unreachable code after this line!\\
\onslide<6->\lstinline!   case File(loc, n, e) => "constructor pattern"!\\
\onslide<7->\lstinline!   case list @ List(0, _, _*) => "sequence pattern"!\\
\onslide<8->\lstinline!   case (x, "test", 10, y) => "tuple pattern"!\\
\onslide<9->\lstinline!   case x: String => "typed pattern"!\\
\onslide<10->\lstinline!   case x: List[String] => "type erasure" // be careful!\\
\onslide<11->\lstinline!   case x: List[_] => "type constructor pattern"!\\
\onslide<2->\lstinline!   case _ => "wildcard pattern"!\\
\onslide<1->\lstinline!}!
\end{frame}

\begin{frame}[fragile]{Kinds of patterns}
\begin{block}{Constant or Variable?}
Identifiers starting with a capital letter or those marked with back ticks are
treated as \highlight{constants}. Any other identifier is treated as a \highlight{variable}.
\end{block}
\pause
\begin{exampleblock}{Constant or Variable?}
\lstinline!val pi = 3.14!\\
\lstinline!!\\
\lstinline!selector match {!\\
\lstinline!   case `pi` => "constant"!\\
\lstinline!   case pi => "variable"!\\
\lstinline!}!
\end{exampleblock}
\end{frame}

\begin{frame}[fragile]{List patterns}
\begin{exampleblock}{List patterns}
\onslide<1->\lstinline!selector match {!\\
\onslide<2->\lstinline!   case Nil => "empty list"!\\
\onslide<3->\lstinline!   case element :: Nil => "a list with exactly one element"!\\
\onslide<4->\lstinline!   case x1 :: x2 :: xs => "a list with at least two elements"!\\
\onslide<5->\lstinline!   case head :: tail => "a list with at least one element"!\\
\onslide<6->\lstinline!   case _ => "not a list"!\\
\onslide<1->\lstinline!}!
\end{exampleblock}
\onslide<7>
\begin{block}{How does it work?}
The \highlight{cons} pattern is a special case of an \highlight{infix} operation
pattern. When seen as a pattern, an infix operation such as \lstinline!p op q!
is equivalent to \lstinline!op(p, q)!. That is, the infix operator
\lstinline!op! is treated as a constructor pattern. So \lstinline!::! is not
only an operator defined on lists, but also a \lstinline!class!
\end{block}
\end{frame}

\begin{frame}[fragile]{List patterns}
\begin{alertblock}{Word of caution}
Avoid mixing cons patterns with ordinary list patterns. Use either
\lstinline!Nil! in combination with \lstinline!head :: tail! or
\lstinline!List()! in combination with \lstinline!List(head, tail @ _*)!, but
not both.
\end{alertblock}
\end{frame}

\section{Case Classes}
\begin{frame}[fragile]{Case Classes}
Case classes are a very convenient way for encoding \highlight{data
structures}\\
\lstinline!!\\
Instead of writing code like this:\\
\lstinline!!\\
\lstinline!class Person(val firstName: String, val lastName: String)!\\
\lstinline!!\\
you can simply write this:\\
\lstinline!!\\
\lstinline!case class Person(firstName: String, lastName: String)!
\end{frame}

\begin{frame}[fragile]{Case Classes}
\begin{block}{Benefits}
\begin{itemize}
  \item \lstinline!firstname! and \lstinline!lastName! are prefixed with \lstinline!val!
  \item \lstinline!object Person! is generated with \lstinline!apply! and \lstinline!unapply!
  \item Implementations of \lstinline!equals!, \lstinline!hashCode! and \lstinline!toString! are generated
\end{itemize}
\end{block}
\pause
\begin{alertblock}{Price}
\begin{itemize}
  \item Memory footprint
  \item Hard dependency between data and its representation
  \item No inheritance between case classes*
\end{itemize}
* Data structures do not need inheritance anyway\ldots
\end{alertblock}
\end{frame}

\begin{frame}[fragile]{Guards}
\lstinline!case class Point(x: Int, y: Int)!\\
\lstinline!!\\
We want to pattern match on points with \lstinline!x == y!\\
\lstinline!!\\
\lstinline!case Point(x, x) => ...!\\
\lstinline!!\\
\alert{error: x is already defined as value x}\\
\lstinline!!\\
\lstinline!case Point(x, y) if x == y => ...!
\end{frame}

\section{Patterns Everywhere}
\begin{frame}[fragile]{Patterns Everywhere - Declarations}
\begin{block}{Patterns in declarations}
Everything between the \lstinline!val! (or \lstinline!var!) and the equals sign
(\lstinline!=!) is a pattern
\end{block}
\begin{lstlisting}
scala> val person = Person("Martin", "Odersky")
person: Person = Person(Martin, Odersky)

scala> val Person(firstName, lastName) = person
firstName: String = Martin
lastName: String = Odersky
\end{lstlisting}
\end{frame}

\begin{frame}[fragile]{Patterns Everywhere - Partial Functions}
\begin{block}{What is a total function?}
A total function is an ordinary function as you know it. It is defined for
\highlight{all} the elements of the set of inputs, specified by its signature.
\end{block}
\pause
\begin{exampleblock}{A total function}
\lstinline!val f: Int => String = _.toString // is defined for all integers!
\end{exampleblock}
\pause
\begin{block}{What is a partial function?}
A partial function is a function in a more general sense. It is defined only for
\highlight{some} elements of the set of inputs, specified by its signature.
\end{block}
\pause
\begin{exampleblock}{A partial function}
\lstinline!val f: Int => String = {!\\
\lstinline!   case 4711 => "4711"!\\
\lstinline!   case x => "not defined for " + x!\\
\lstinline!} // is defined only for 4711!
\end{exampleblock}
\end{frame}

\begin{frame}[fragile]{Patterns Everywhere - Partial Functions}
\begin{alertblock}{Reminder}
Scala is a statically typed language. Therefore, the type checker
\highlight{verifies} proofs at compile time. These proofs are about
\highlight{types}, not \alert{values}. If the partial function is not defined
for the value it receives at runtime, an exception is thrown.
\end{alertblock}
\end{frame}

\begin{frame}[fragile]{Patterns Everywhere - Partial Functions}
\begin{block}{Partial where total is required}
A partial function can be used \highlight{anywhere} a function literal can be
used
\end{block}
\lstinline!!\\
\onslide<2->\lstinline!val capitals = Map("England" -> "London", "Germany" -> "Berlin")!\\
\onslide<3->\lstinline!// Instead of using a total function!\\
\onslide<3->\lstinline!capitals foreach { tuple => !\\
\onslide<3->\lstinline!   println("The capital of " + tuple._1 + " is " + tuple._2)!\\
\lstinline!}!\\
\onslide<4->\lstinline!// or a total function with a match!\\
\onslide<4->\lstinline!capitals foreach { tuple => tuple match { case (country, city) =>!\\
\onslide<4->\lstinline!   println("The capital of " + tuple._1 + " is " + tuple._2)!\\
\lstinline!}}!\\
\onslide<5->\lstinline!// you can use a partial function directly!\\
\onslide<5->\lstinline!capitals foreach { case (country, city) =>!\\
\onslide<5->\lstinline!   println("The capital of " + country + " is " + city)!\\
\lstinline!}!
\end{frame}

\begin{frame}[fragile]{Patterns Everywhere - Partial Functions}
\begin{exampleblock}{\lstinline!collect! is a combination of \lstinline!filter! and \lstinline!map!}
\begin{lstlisting}
scala> val objects = List(1, "Hello", 2, "World", 3)
objects: List[Any] = List(1, Hello, 2, World, 3)

scala> val words = objects collect { case s: String => s }
words: List[String] = List(Hello, World)

scala> val greeting = words mkString " "
greeting: String = Hello World
\end{lstlisting}
\end{exampleblock}
\end{frame}

\begin{frame}[fragile]{Patterns Everywhere - for comprehensions}
\begin{block}{for comprehensions are discussed in detail in Part 11 - Collections}
\begin{lstlisting}
for ((country, city) <- capitals)
   println("The capital of " + country + " is " + city)
\end{lstlisting}
\end{block}
\end{frame}

\begin{frame}[fragile]{Patterns Everywhere - Exceptions}
\begin{lstlisting}
try bungeeJumping()
catch {
  case oops: CordTooLongException => ...
  case scary: TooHighException => ...
  case e: Exception => ...
} finally showDeadOrAlive()
\end{lstlisting}
\end{frame}

\begin{frame}[fragile]{Patterns Everywhere - Exceptions}
\begin{lstlisting}
try bungeeJumping()
catch meIfYouCan
finally showDeadOrAlive()

val meIfYouCan: Throwable => Unit = {
  case oops: CordTooLongException => ...
  case scary: TooHighException => ...
  case e: Exception => ...
}
\end{lstlisting}
\end{frame}

\begin{frame}[fragile]{Patterns Everywhere - Regex}
\begin{lstlisting}
scala> val Date = """(\d\d)/(\d\d)/(\d\d\d\d)""".r
Date: scala.util.matching.Regex = (\d\d)/(\d\d)/(\d\d\d\d)

scala> val sample = "16/05/2012"
sample: String = 16/05/2012

scala> val Date(day, month, year) = sample
day: String = 16
month: String = 05
year: String = 2012
\end{lstlisting}
\end{frame}

\section{Option Cheat Sheet}
\begin{frame}[fragile]{Option Cheat Sheet}
\begin{alertblock}{\lstinline!def flatMap[B](f: (A) => Option[B]): Option[B]!}
\begin{lstlisting}
option match {
   case None => None
   case Some(x) => foo(x)
}
\end{lstlisting}
\end{alertblock}
\begin{exampleblock}{\lstinline!def flatMap[B](f: (A) => Option[B]): Option[B]!}
\lstinline!option flatMap foo!
\end{exampleblock}
\end{frame}

\begin{frame}[fragile]{Option Cheat Sheet}
\begin{alertblock}{\lstinline!def flatten[CC[_], B]: CC[B]!}
\begin{lstlisting}
option match {
   case None => None
   case Some(x) => x
}
\end{lstlisting}
\end{alertblock}
\begin{exampleblock}{\lstinline!def flatten[CC[_], B]: CC[B]!}
\lstinline!option.flatten!
\end{exampleblock}
\end{frame}

\begin{frame}[fragile]{Option Cheat Sheet}
\begin{alertblock}{\lstinline!def map[B](f: A => B): Option[B]!}
\begin{lstlisting}
option match {
   case None => None
   case Some(x) => Some(foo(x))
}
\end{lstlisting}
\end{alertblock}
\begin{exampleblock}{\lstinline!def map[B](f: A => B): Option[B]!}
\lstinline!option map foo!
\end{exampleblock}
\end{frame}

\begin{frame}[fragile]{Option Cheat Sheet}
\begin{alertblock}{\lstinline!def foreach[U](f: A => U): Unit!}
\begin{lstlisting}
option match {
   case None => {}
   case Some(x) => foo(x)
}
\end{lstlisting}
\end{alertblock}
\begin{exampleblock}{\lstinline!def foreach[U](f: A => U): Unit!}
\lstinline!option foreach foo!
\end{exampleblock}
\end{frame}

\begin{frame}[fragile]{Option Cheat Sheet}
\begin{alertblock}{\lstinline!def isDefined: Boolean!}
\begin{lstlisting}
option match {
   case None => false
   case Some(_) => true
}
\end{lstlisting}
\end{alertblock}
\begin{exampleblock}{\lstinline!def isDefined: Boolean!}
\lstinline!option.isDefined!
\end{exampleblock}
\end{frame}

\begin{frame}[fragile]{Option Cheat Sheet}
\begin{alertblock}{\lstinline!def isEmpty: Boolean!}
\begin{lstlisting}
option match {
   case None => true
   case Some(_) => false
}
\end{lstlisting}
\end{alertblock}
\begin{exampleblock}{\lstinline!def isEmpty: Boolean!}
\lstinline!option.isEmpty!
\end{exampleblock}
\end{frame}

\begin{frame}[fragile]{Option Cheat Sheet}
\begin{alertblock}{\lstinline!def forall(pred: A => Boolean): Boolean!}
\begin{lstlisting}
option match {
   case None => true
   case Some(x) => foo(x)
}
\end{lstlisting}
\end{alertblock}
\begin{exampleblock}{\lstinline!def forall(pred: A => Boolean): Boolean!}
\lstinline!option forall foo!
\end{exampleblock}
\end{frame}

\begin{frame}[fragile]{Option Cheat Sheet}
\begin{alertblock}{\lstinline!def exists(p: A => Boolean): Boolean!}
\begin{lstlisting}
option match {
   case None => false
   case Some(x) => foo(x)
}
\end{lstlisting}
\end{alertblock}
\begin{exampleblock}{\lstinline!def exists(p: A => Boolean): Boolean!}
\lstinline!option exists foo!
\end{exampleblock}
\end{frame}

\begin{frame}[fragile]{Option Cheat Sheet}
\begin{alertblock}{\lstinline!def orElse[B >: A](alternative: => Option[B]): Option[B]!}
\begin{lstlisting}
option match {
   case None => foo
   case Some(x) => Some(x)
}
\end{lstlisting}
\end{alertblock}
\begin{exampleblock}{\lstinline!def orElse[B >: A](alternative: => Option[B]): Option[B]!}
\lstinline!option orElse foo!
\end{exampleblock}
\end{frame}

\begin{frame}[fragile]{Option Cheat Sheet}
\begin{alertblock}{\lstinline!def getOrElse[B >: A](default: => B): B!}
\begin{lstlisting}
option match {
   case None => foo
   case Some(x) => x
}
\end{lstlisting}
\end{alertblock}
\begin{exampleblock}{\lstinline!def getOrElse[B >: A](default: => B): B!}
\lstinline!option getOrElse foo!
\end{exampleblock}
\end{frame}

\begin{frame}[fragile]{Option Cheat Sheet}
\begin{alertblock}{\lstinline!def toList: List[A]!}
\begin{lstlisting}
option match {
   case None => Nil
   case Some(x) => x :: Nil
}
\end{lstlisting}
\end{alertblock}
\begin{exampleblock}{\lstinline!def toList: List[A]!}
\lstinline!option.toList!
\end{exampleblock}
\end{frame}

\section{Summary}
\begin{frame}{Summary}
\begin{itemize}
  \item Objects and Data Structures are \highlight{diametrically opposed}
  \item \lstinline!match! \highlight{expressions} are very powerful
  \item Extractors help you to \highlight{extract} data out of data structures
  \item Extractors are objects with \highlight{def unapply} or \highlight{def unapplySeq}
  \item \highlight{Tuples} are heterogeneous collections with fixed size
  \item \highlight{Option} is a better way of handling \alert{null}
  \item The \highlight{order} of case alternatives is significant
  \item Pattern matching is \highlight{deep}
  \item \highlight{Case classes} should only be used as pure data structures
  \item Pattern matching is \highlight{all over Scala}
  \item Partial functions can be used \highlight{anywhere} a lambda can be used  
\end{itemize}
\end{frame}