\part[The Java Ecosystem]{The Java Ecosystem}
\section{Fight metal programming}

\begin{frame}{Once upon a time\ldots}
\visible<2->{in the land of computers there were 10101110010101101\ldots\\}
\visible<3->{then assemblers came along\ldots\\}
\visible<4->{followed by compilers\ldots\\}
\visible<5->{and they lived happily ever after\ldots \visible<6->{\alert{NOT!}}}
\end{frame}

\begin{frame}[fragile]{Portability}
\begin{alertblock}{Platform Dependency}
\begin{lstlisting}
for(os <- List("linux", "mac", "windows"))
  println("If you compile a C program under " + os
            + "you can run it only on " + os + ".")
\end{lstlisting}
\end{alertblock}
\pause
Sometimes a recompilation is not sufficient. \alert{The code} has to be adapted!
\pause
\begin{block}{Platform Independency}
Write Once, Run Anywhere.
\end{block}
\end{frame}

\pictureframe{Problem - Unmanaged code}{resources/PlatformDependency.pdf}

\begin{frame}{Problem - Unmanaged code 2}
\begin{block}{Why is unmanaged code a problem?}
\begin{enumerate}
  \pause
  \item portability issues
  \pause
  \item low level hazards like buffer overflows, segmentation faults etc.
\end{enumerate}
\end{block}
\pause
\begin{alertblock}{The problem, programming language designers are trying to
solve:} Software development is not about programming the hardware, it is about
abstracting over the hardware!
\end{alertblock}
\end{frame}

\pictureframe{Solution - Managed code}{resources/PlatformIndependency.pdf}

\begin{frame}{Solution - Managed code 2}
\begin{block}{Why is managed code a solution?}
\begin{enumerate}
  \pause
  \item portability issues are being taken care by VM vendors.
  \pause
  \item low level hazards like buffer overflows, segmentation faults etc. are
  being handled by the VM.
  \pause
  \item programs as-well as programming languages are changing every second;
  platforms, are fairly stable!
\end{enumerate}
\end{block}
\pause
\begin{block}{Wind of Change}
\begin{itemize}
  \item it is \textcolor{blue}{feasible} to change a program.
  \item it is \textcolor{orange}{hard} to change a programming language.
  \item it is \textcolor{red}{very hard} to change a platform, but platforms are
  not required to be frequently changed.
\end{itemize}
\end{block}
\end{frame}

\begin{frame}{Virtual Machines}
\begin{block}{Operating System}
An operating system is an abstraction over a hardware \highlight{machine}.
\end{block}
\pause
\begin{block}{Virtual Machine}
A \highlight{virtual} machine is an abstraction over an operating system.
\end{block}
\pause
\begin{exampleblock}{Well-known VMs}
\begin{description}
  \item[Java] Java Virtual Machine (JVM)
  \item[.Net] Common Language Runtime (CLR)
  \item[Android] Dalvik virtual machine
  \item
  and\link{http://www.en.wikipedia.org/wiki/Virtual_machine\#List_of_virtual_machine_software}{many more}
\end{description}
\end{exampleblock}
\end{frame}

\begin{frame}{Managed Programming Languages}
\begin{block}{JVM - runs everywhere}
\begin{description}[<+->]
	\item[\link{http://en.wikipedia.org/wiki/Java_\%28programming_language\%29}{Java}]
	Statically typed OO language
	\item[\link{http://en.wikipedia.org/wiki/Scala_programming_language}{Scala}]
	Statically typed OO \& FP language
	\item[\link{http://en.wikipedia.org/wiki/Clojure}{Clojure}] Dynamically
	typed FP language
	(\link{http://en.wikipedia.org/wiki/Lisp_\%28programming_language\%29}{Lisp}dialect)
	\item[\link{http://en.wikipedia.org/wiki/Groovy_\%28programming_language\%29}{Groovy}]
	Dynamically typed scripting language
	\item and\link{http://en.wikipedia.org/wiki/List_of_JVM_languages}{many
	more}
\end{description}
\end{block}
\begin{block}{CLR - runs only on Windows}
\begin{description}[<+->]
	\item[\link{http://en.wikipedia.org/wiki/C_Sharp_\%28programming_language\%29}{C\#}]
	Statically typed OO \& FP language
	\item[\link{http://en.wikipedia.org/wiki/Visual_Basic_.NET}{VB}] Statically
	typed OO language
	\item[\link{http://en.wikipedia.org/wiki/F_Sharp_\%28programming_language\%29}{F\#}]
	Statically typed FP language
	\item
	and\link{http://en.wikipedia.org/wiki/List_of_CLI_languages\#CLI_languages}{many more}
\end{description}
\end{block}
\end{frame}

\begin{frame}{Managed Programming Languages 2}
\begin{block}{Dalvik - runs only on Android}
Java is officially supported. Every JVM language should theoretically work.\highlight{Scala}
works.\link{http://stackoverflow.com/questions/1994703/which-programming-languages-can-i-use-on-android-dalvik}{Few others} work as-well.
\end{block}
\end{frame}

\begin{frame}{JVM - People's Choice}
\begin{center}
\includegraphics[scale=0.25]{resources/JVM.png}
\end{center}
The code is \alert{not} compiled to \highlight{binary}.\\
The code is compiled to the so called \highlight{byte-code}.\\
The \highlight{JVM} is a software component, which runs the
\highlight{byte-code}.\\
A JVM implementation exists for almost \highlight{every platform}.\\
\highlight{Java} is the most used language on the \highlight{JVM}.\\
The most used \highlight{Java} compiler is called \highlight{javac}. 
\end{frame}

\pictureframe{What language should we use for the JVM?}{resources/Language.pdf}
\pictureframe{Can we reuse the java compiler for this
language?}{resources/Compiler.pdf}

\section{A History Lesson}

\part[Introducing Scala]{Introducing Scala}
\part[Programming Paradigms]{Programming Paradigms}
\part[Scala's Development Environment]{Scala's Development Environment}
\part[Programming Styles]{Programming Styles}
\part[Higher-Order Functions]{Higher-Order Functions}
\part[Growing The Language]{Growing The Language}
\part[TDD \& Clean Code]{Test-driven Development \& Clean Code}
\part[Pattern Matching]{Pattern Matching}
\part[Traits]{Traits}
\part[Collections]{Collections}
\part[Concurrency]{Concurrency}
\part[DSLs]{Domain Specific Languages}
\part[Higher-Kinded Types]{Higher-Kinded Types}
