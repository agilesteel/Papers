\part[Introducing Scala]{Introducing Scala}
\section{Scalable Language}
\begin{frame}{Scalable Language}
\begin{center}
\includegraphics{resources/Scala.jpg}
\end{center}
\begin{center}
Scala stands for ``\highlight{sca}lable \highlight{la}nguage''.\\
Scala translates from Latin to English as ``\highlight{stairs}''.\\
Scala's logo shows a circular staircase.
\end{center}
\end{frame}

\begin{frame}{Scalable Language}
Scala is scalable in a sense of:
\begin{description}
  \item[Syntax:] growing with the demands of its users
  \item[FP:] hacking small scripts
  \item[OO:] developing enterprise applications
\end{description}
\end{frame}

\begin{frame}[fragile]{Growing the Language}
If you were in need of a \lstinline!type! of \lstinline!Polynomial! in Java you
would write it yourself and end up with something like this:
\begin{exampleblock}{Java}
\begin{lstlisting}[language=java]
Polynomial sum = firstPolynomial.plus(secondPolynomial)
\end{lstlisting}
\end{exampleblock}
In Scala the code would look like this:
\begin{exampleblock}{Scala}
\begin{lstlisting}
val sum = firstPolynomial + secondPolynomial
\end{lstlisting}
\end{exampleblock}
Your library would feel as if it was in the language from the beginning.
\end{frame}

\begin{frame}[fragile]{Small Scripts}
\begin{exampleblock}{Run scripts on windows (Paste this into a \highlight{.bat}
or \highlight{.cmd} file)}
\begin{verbatim}
::#! 
@echo off 
call scala \%0 \%* 
goto: eof 
::!#
\end{verbatim}
\lstinline!// The script starts here!
\end{exampleblock}
\begin{exampleblock}{Run scripts on unix}
\begin{verbatim}
#!/bin/sh
exec scala "$0" "$@"
!#
\end{verbatim}
\lstinline!// The script starts here!
\end{exampleblock}
\end{frame}

\begin{frame}[fragile]{Small Scripts}
\begin{exampleblock}{Get title of the Scala page on wiki}
\begin{lstlisting}
def title = xmlTitle.text
def xmlTitle = scala.xml.XML loadString stringTitle.get
def stringTitle = lines find { _ startsWith "<title>" }
def lines = io.Source.fromURL(url).getLines
def url = wiki + scalaLang
def wiki = "http://en.wikipedia.org/wiki/"
def scalaLang = "Scala_(programming_language)"

println(title) // Scripts must end with an expression
\end{lstlisting}
\end{exampleblock}
\end{frame}

\pictureframe{Enterprise applications}{resources/ScalaIndustry.pdf}

\section{Syntax}
\begin{frame}[fragile]{Reading Scala}
\begin{block}{The types are specified next to an identifier separated by a colon}
\begin{tabular}{l|l}
\textsc{Scala} & \textsc{Java}\\
\hline
\lstinline!x: Int! & \lstinline!int x!\\
\lstinline!getBeerPrice(): Int! & \lstinline!int getBeerPrice()!\\
\lstinline!setBeerPrice(price: Int): Unit! & \lstinline!void setBeerPrice(int price)!\\
\end{tabular}
\end{block}
\begin{block}{Only 3 keywords to remember and you are ready to go}
\begin{tabular}{ll}
\lstinline!var! & declares a variable\\
\lstinline!val! & declares a value (immutable variable)\\
\lstinline!def! & defines a function \\
\end{tabular}
\end{block}
\end{frame}

\begin{frame}[fragile]{Reading Scala}
\begin{exampleblock}{First Acquaintance}
\begin{lstlisting}
scala> val pi: Double = 3.14;
pi: Double = 3.14  // interpreter response

scala> def calculateCircumference(r: Int):Double = {
     |   return 2 * pi * r;
     | }
calculateCircumference: (r: Int)Double // interpreter response

scala> var circ: Double = calculateCircumference(10);
circ: Double = 62.800000000000004 // interpreter response
\end{lstlisting}
\end{exampleblock}
\end{frame}

\begin{frame}[fragile]{Writing Scala}
\begin{exampleblock}{Second opinion}
\begin{lstlisting}
val pi = 3.14
def calculateCircumference(r: Int) = 2 * pi * r
var circ = calculateCircumference(10)
\end{lstlisting}
\end{exampleblock}
\begin{itemize}
  \item semicolons are inferred
  \item most types are inferred
  \item last expression is implicitly returned
  \item single expression in the body means no need for curly braces
\end{itemize}
\end{frame}

\begin{frame}{What does ``static typing'' mean?}

\end{frame}

\begin{frame}{What does ``immutable'' stand for?}

\end{frame}

\begin{frame}{``Everything'' is an expression}

\end{frame}

\begin{frame}{What is not an expression?}

\end{frame}