\part[The Scala Ecosystem]{The Scala Ecosystem}
\section{The Web}
\begin{frame}{Official Scala Website}
\begin{center}
\link{http://www.scala-lang.org/}{http://www.scala-lang.org/}
\end{center}
Stuff you find there:
\begin{itemize}
  \item Scala download link
  \item Documentation
  \item Guides
  \item Support information
  \item News
  \item and so on\ldots
\end{itemize}
\end{frame}

\begin{frame}{Official Industry-oriented Website}
\begin{center}
\link{http://www.typesafe.com/}{http://www.typesafe.com/}
\end{center}
Typesafe is a combination of:
\begin{description}
\item[\link{http://www.scala-lang.org/}{Scala}] programming language
\item[\link{http://akka.io/}{Akka}] event-driven middleware
\item[\link{http://www.playframework.org/}{Play!}] web framework
\end{description}
\pause
Typesafe provides \link{http://www.typesafe.com/stack}{The Typesafe Stack}:
\begin{itemize}
\item Scala, Akka, Play!
\item Simple Build Tool (\link{http://www.scala-sbt.org/}{SBT})
\item \link{https://github.com/n8han/giter8}{giter8} (a command line tool to
apply templates defined on github)
\end{itemize}
\end{frame}

\begin{frame}{Official Community-driven Documentation Website}
\begin{center}
\link{http://docs.scala-lang.org/}{http://docs.scala-lang.org/}
\end{center}
Stuff you find there:
\begin{itemize}
  \item Guides
  \item Tutorials
  \item Glossary
  \item Cheat sheets
  \item Scala Improvement Process
  \item \link{http://docs.scala-lang.org/style/}{Style Guide}
  \item and so on\ldots
\end{itemize}
\end{frame}

\begin{frame}{Scala Wiki}
\begin{center}
\link{https://wiki.scala-lang.org}{https://wiki.scala-lang.org}
\end{center}
Stuff you find there:
\begin{itemize}
  \item \link{https://wiki.scala-lang.org/display/SW/Tools+and+Libraries}{Tools
  and Libraries}
  \item Twitter links
  \item User Groups
\end{itemize}
\end{frame}

\begin{frame}{Stack Overflow}
\begin{center}
\link{http://stackoverflow.com}{http://stackoverflow.com}
\end{center}
Stack Overflow is a question \& answer site about programming questions.\\
Stuff you find / can do there:
\begin{itemize}
  \item \link{http://stackoverflow.com/tags/scala/info}{Stack Overflow Scala Tutorial!}
  \item search for questions
  \item ask questions
  \item answer questions ;)
  \item follow topics of interest
  \item earn reputation
\end{itemize}
\end{frame}

\pictureframe{Scala on Stack Overflow and Github}{resources/SOGithub.pdf}

\begin{frame}{Symbol Hound}
\begin{center}
\link{http://symbolhound.com/}{http://symbolhound.com/}
\end{center}
\begin{center}
\emph{Symbol Hound} is a search engine for symbols.\\
Scala has lots of symbols.\\
Very \highlight{few} of them are language \highlight{keywords}.
\end{center}
\end{frame}

\begin{frame}{Simply Scala}
\begin{center}
\link{http://www.simplyscala.com/}{http://www.simplyscala.com/}
\end{center}
\begin{center}
\emph{Simply Scala} is an \highlight{interactive} Scala tutorial.
\end{center}
\end{frame}

\section{IDEs}
\begin{frame}{The old school way}
This is how you get the ``stand-alone'' Scala edition:
\begin{enumerate}
  \item Go to
  \link{http://www.scala-lang.org/downloads}{http://www.scala-lang.org/downloads}
  \item Download one of the \highlight{Stable} Releases
  \item Extract the files from the archive
  \item Manually adjust the environment variable for the \highlight{Path} to
  point to the \highlight{ScalaHomeDirectory/bin} folder
\end{enumerate}
\end{frame}

\begin{frame}{The old school way}
This is what you get from the ``stand-alone'' Scala edition:
\begin{description}
  \item[scala] Scala's interactive shell (REPL)
  \item[scalac] The Scala compiler
  \item[fsc] The Fast Scala compiler (scalac, which always runs as a daemon in
  the background)
  \item[sbaz] The Scala Bazar (a small console application, which provides
  access to main Scala repositories)
\end{description}
\begin{center}
Start the terminal and type ``\highlight{sbaz help}'' to get started
\end{center}
\end{frame}

\begin{frame}{The eclipse Scala plugin}
\begin{center}
\link{http://scala-ide.org/}{http://scala-ide.org/}
\end{center}
\begin{center}
An extra Scala installation is \highlight{not} required for the Scala
plugin for eclipse to work!
\end{center}
The plugin gives you:
\begin{enumerate}
  \item Support for Mixed Scala/Java Projects
  \item Scala perspective
  \item Scala compiler
  \item Scala REPL, which runs inside of Eclipse
  \item Scala debugger
\end{enumerate}
\end{frame}

\begin{frame}{The IntelliJ IDEA Scala plugin}
\begin{center}
\link{http://blog.jetbrains.com/scala/}{http://blog.jetbrains.com/scala/}
\end{center}
\begin{center}
An extra Scala installation \alert{is} required for the Scala
plugin for IntelliJ IDEA to work!
\end{center}
\begin{center}
Get started \link{http://confluence.jetbrains.net/display/SCA/Getting+Started+with+IntelliJ+IDEA+Scala+Plugin}{here}
\end{center}
\end{frame}

\begin{frame}{Simple Build Tool}
\begin{center}
\link{http://www.scala-sbt.org/}{http://www.scala-sbt.org/}
\end{center}
\begin{center}
Get started
\link{https://github.com/harrah/xsbt/wiki/Getting-Started-Setup}{here}
\end{center}
SBT gives you:
\begin{enumerate}
  \item Scala REPL
  \item Scala-scripted configuration luxury
  \item \highlight{Continous incremental compilation}
  \item Out of the box testing with Scala test frameworks
  \item Concise dependency management DSL
  \item and \link{http://www.scala-sbt.org/}{further benefits}
\end{enumerate}
\end{frame}

\begin{frame}{The Typesafe Stack}
\begin{center}
\link{http://www.typesafe.com/stack/download}{http://www.typesafe.com/stack/download}
\end{center}
\begin{center}
Get started \link{http://www.typesafe.com/resources/getting-started/}{here}
\end{center}
\end{frame}

\begin{frame}{Recommended setup}
\begin{enumerate}
  \item Install the JVM
  \item Install eclipse
  \item Install the scala plugin for eclipse
  \item Install the typesafe stack
\end{enumerate}
The only reason to install the stand-alone Scala edition is the REPL, but the
REPL can be accessed from either eclipse or sbt, which is more than enough.
\end{frame}