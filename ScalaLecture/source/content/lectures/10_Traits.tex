\part[Traits]{Traits}
\begin{frame}{Traits}
Code reuse is very important to all applications, especially to large ones like
in the enterprise. Scala gives you functional constructs and pushes the OO
paradigm towards its ideal. Functional paradigm shines with
\highlight{higher-order functions}. \highlight{Inheritance} is in the spotlight
of the object-oriented programming. But inheritance is known to be very
\alert{fragile}. Most programming books encourage the usage of composition over inheritance.
Joshua Blochs famous book \highlight{``Effective Java''} devotes a whole chapter
to this topic called \highlight{``Favor composition over inheritance''}. In a
way, Scala enables you to \highlight{express composition through inheritance}!
This concept is called \highlight{``mixin composition''}.
\end{frame}

\section{Interfaces}
\subsection{Thin Interfaces}
\begin{frame}[fragile]{Thin Interfaces}
\begin{block}{Thin Interfaces}
Works exactly the same way as in Java, but instead of the
\lstinline!implements! keyword, the \lstinline!with! keyword is
used.
\end{block}
\pause
\begin{exampleblock}{Thin Interfaces}
\onslide<2->\lstinline!class Animal!\\
\onslide<3->\lstinline!trait Pet {!\\
\onslide<3->\lstinline!   def allowedToSleepInBed: Boolean!\\
\onslide<3->\lstinline!}!\\
\onslide<4->\lstinline!class Cat extends Animal with Pet {!\\
\onslide<4->\lstinline!   val allowedToSleepInBed = true!\\
\onslide<4->\lstinline!}!\\
\onslide<5->\lstinline!class Turtle extends Animal with Pet {!\\
\onslide<5->\lstinline!   def allowedToSleepInBed = false!\\
\onslide<5->\lstinline!}!
\end{exampleblock}
\end{frame}
\subsection{Rich Interfaces}
\begin{frame}{Rich Interfaces}
\begin{center}
\begin{tabular}{|l|cc|}
\hline
& Concrete members & Abstract members\\
\hline
Java's concrete classes & \highlight{yes} & \alert{no}\\
Java's abstract classes & \highlight{yes} & \highlight{yes}\\
Java's interfaces & \alert{no} & \highlight{yes}\\
\hline
\hline
Scala's concrete classes & \highlight{yes} & \alert{no}\\
Scala's abstract classes & \highlight{yes} & \highlight{yes}\\
Scala's traits & \textcolor{green}{yes} & \highlight{yes}\\
\hline
\end{tabular}
\end{center}
\end{frame}

\begin{frame}[fragile]{Rich Interfaces}
\onslide<1->\lstinline!class Animal!\\
\onslide<1->\lstinline!trait Pet {!\\
\onslide<1->\lstinline!   def allowedToSleepInBed: Boolean!\\
\onslide<2->\lstinline{   def disallowedToSleepInBed: Boolean = !allowedToSleepInBed}\\
\onslide<1->\lstinline!}!\\
\onslide<1->\lstinline!class Cat extends Animal with Pet {!\\
\onslide<1->\lstinline!   val allowedToSleepInBed = true!\\
\onslide<1->\lstinline!}!\\
\onslide<1->\lstinline!class Turtle extends Animal with Pet {!\\
\onslide<1->\lstinline!   def allowedToSleepInBed = false!\\
\onslide<1->\lstinline!}!
\end{frame}

\begin{frame}[fragile]{Rich Interfaces}
\begin{block}{Rich interface \lstinline!Ordered! from the Scala library}
The only thing the \highlight{implementers} need to do is implement the \lstinline!def compare(that: A): Int!.
The \highlight{clients} on the other hand have the convenience of choosing
between \highlight{4 additional methods}.
\end{block}

\begin{block}{Rich interface \lstinline!Ordered! from the Scala library}
\begin{lstlisting}
trait Ordered[A] {
   def compare(that: A): Int
   
   def <  (that: A): Boolean = (this compare that) <  0
   def >  (that: A): Boolean = (this compare that) >  0
   def <= (that: A): Boolean = (this compare that) <= 0
   def >= (that: A): Boolean = (this compare that) >= 0
}
\end{lstlisting}
\end{block}
\end{frame}

\section{Stackable Modifications}
\begin{frame}{Stackable Modifications}
Traits are much more powerful than simple interfaces. In fact traits act as
interfaces almost ``by accident''. Scala is a very minimalistic language. Instead
of introducing ``new'' keywords like \lstinline!interface! or
\lstinline!implements! Scala designers decided to make traits do all the work.
\end{frame}

\subsection{Adding Functionality}
\begin{frame}[fragile]{Stackable Modifications - Adding Functionality}
Imagine you want to log the creation time of your objects. In order to do that you
define the \lstinline!class Timestamp!. This class
introduces a single member \lstinline!val creationTime! which calls a private
member \lstinline!def currentTime! as soon as an instance of
\lstinline!Timestamp! is constructed. So the only thing you can do with objects
of this \lstinline!class! is ask them when they were created.
\end{frame}

\begin{frame}[fragile]{Stackable Modifications - Adding Functionality}
\begin{lstlisting}
class Timestamp {
   val creationTime = currentTime
   
   private def currentTime = {
      import java.util.Calendar._
      val calendar = getInstance
      def hours = calendar.get(HOUR_OF_DAY)
      def minutes = calendar.get(MINUTE)
      def seconds = calendar.get(SECOND)
      def milliseconds = calendar.get(MILLISECOND)
      
      hours + ":" + minutes + ":" + seconds + ":" + milliseconds
   }
}
\end{lstlisting}
\end{frame}

\begin{frame}[fragile]{Stackable Modifications - Adding Functionality}
Now imagine you have a \lstinline!class Message! and you want to add the
functionality of the \lstinline!Timestamp! to it. What options do we have and
what design decision will we go with?
\end{frame}

\begin{frame}[fragile]{Stackable Modifications - Adding Functionality}
\begin{alertblock}{Inheritance}
Inheritance is of course the wrong way for three obvious reasons. First, a
\lstinline!Message! \emph{is not} a \lstinline!Timestamp!, which is a clear indication against
inheritance. Second, due to Java's single inheritance policy you will not be
able to extend any other class. You can either extend \lstinline!Timestamp! or say
\lstinline!AbstractMessage!, but not both! The third problem occurs when the
\lstinline!class Message! is provided by a third party. So you can't even dream
about modifying it.
\end{alertblock}
\end{frame}

\begin{frame}[fragile]{Stackable Modifications - Adding Functionality}
\begin{block}{Composition}
Composition is the right way, but it still has one problem and one potential
inconvenience. The ``third party'' issue is obviously the problem, which
can neither be solved by inheritance nor by composition alone.
The potential inconvenience, on the other hand, is far from obvious to Java
programmers. Java's composition concept \alert{does not define a type}! This
issue should not be taking lightly. The flexibility of the indirect transfer of
control and static type checking are sacrificed literally for nothing!
\end{block}
\end{frame}

\begin{frame}[fragile]{Stackable Modifications - Adding Functionality}
\begin{block}{Composition through Inheritance}
Inheritance and composition together might do the trick. Suppose
\lstinline!Message! is provided by a third party and is not \lstinline!final!!
Now you can extend it with your own \lstinline!class MessageWithTimestamp! and
add the \lstinline!Timestamp! via composition. \alert{The type still gets lost
though}. Thus methods which take \lstinline!Timestamp! cannot be invoked with
\lstinline!MessageWithTimestamp!.
\end{block}
\end{frame}

\begin{frame}[fragile]{Stackable Modifications - Adding Functionality}
\begin{block}{Mixin Composition}
Scala, as well as Java \alert{does not support multiple inheritance}. Thus a
class can only extend a \highlight{single type} (Since traits can be
used to defined types, the type our class extends from might be implemented as a trait itself).
Good news: there is another way of using a trait! Instead of
\highlight{extending} a trait you can choose to \highlight{mix} it
\highlight{into} a class (or another trait for that matter).
\end{block}
\pause
\begin{block}{Mixin Composition}
\lstinline!// Thus there is a difference between:!\\
\lstinline!class A extends B!\\
\lstinline!// and!\\
\lstinline!class A extends SomethingElse with B!\\
\lstinline!// where B is a trait.!
\end{block}
\end{frame}

\begin{frame}[fragile]{Stackable Modifications - Adding Functionality}
\begin{lstlisting}
trait Timestamp { // Timestamp is a trait now
   val creationTime = currentTime
   
   private def currentTime = {
      import java.util.Calendar._
      val calendar = getInstance
      def hours = calendar.get(HOUR_OF_DAY)
      def minutes = calendar.get(MINUTE)
      def seconds = calendar.get(SECOND)
      def milliseconds = calendar.get(MILLISECOND)
      
      hours + ":" + minutes + ":" + seconds + ":" + milliseconds
   }
}
\end{lstlisting}
\end{frame}

\begin{frame}[fragile]{Stackable Modifications - Adding Functionality}
\begin{block}{Mixin Composition}
\begin{itemize}
   \item The type is preserved thanks to \highlight{inheritance}
   \item The new functionality is gained thanks to \highlight{composition}
   \item The third party issue is resolved thanks to \highlight{linearization}
\end{itemize}
\end{block}
\begin{exampleblock}{Mixin Composition}
\onslide<1->\lstinline!// static* mixin!\\
\onslide<1->\lstinline!class MyMessage extends com.thirdparty.Message with Timestamp!\\
\onslide<2->\lstinline!// dynamic* mixin!\\
\onslide<2->\lstinline!val message = new com.thirdparty.Message with Timestamp!\\
\onslide<3->\lstinline!showCreationTimeOf(message)!\\
\onslide<3->\lstinline!def showCreationTimeOf(some: Timestamp) {!\\
\onslide<3->\lstinline!   println(some.creationTime)!\\
\onslide<3->\lstinline!}!\\
\end{exampleblock}
\onslide<2->* static/dynamic should not be confused with compile time/runtime here.
\end{frame}

\subsection{Modifying Functionality}
\pictureframe{Stackable Modifications - Modifying Functionality}{resources/Stackable.pdf}

\begin{frame}{Stackable Modifications - Modifying Functionality}
\begin{center}
We want to write a racing game ;)
\end{center}
\end{frame}

\begin{frame}[fragile]{Stackable Modifications - Modifying Functionality}
\begin{exampleblock}{Base}
\begin{lstlisting}
abstract class Car {
   def model: String
   def topSpeed: Int
   def maxAcceleration: Int
   override def toString = {
      def brand = getClass.getSimpleName
      brand + " " + model + " " + topSpeed + " " + maxAcceleration
   }
}
\end{lstlisting}
\end{exampleblock}
\end{frame}

\begin{frame}[fragile]{Stackable Modifications - Modifying Functionality}
\begin{exampleblock}{Core(s)}
\begin{lstlisting}
class OrdinaryCar(override val model: String) extends Car {
   def topSpeed = 220 // km/h
   def maxAcceleration = 8000 // rpm
}

class SportsCar(override val model: String) extends Car {
   def topSpeed = 300 // km/h
   def maxAcceleration = 11000 // rpm
}
\end{lstlisting}
\end{exampleblock}
\end{frame}

\begin{frame}[fragile]{Stackable Modifications - Modifying Functionality}
\begin{exampleblock}{Stackable Modifications}
\begin{lstlisting}
trait Spoiler extends Car {
   abstract override def topSpeed =
      (super.topSpeed * 1.02).toInt
}
\end{lstlisting}
\pause
\begin{lstlisting}
trait EngineControlUnit extends OrdinaryCar {
   abstract override def topSpeed = 
      (super.topSpeed * 1.5).toInt
}
\end{lstlisting}
\pause
\begin{lstlisting}
trait TurboCharger extends OrdinaryCar {
   abstract override def maxAcceleration =
      (super.maxAcceleration * 1.25).toInt
}
\end{lstlisting}
\end{exampleblock}
\end{frame}

\begin{frame}[fragile]{Stackable Modifications - Modifying Functionality}
Each trait is a one-liner, but has so much going on. If a trait extends a type,
it will become a modification trait. It declares \highlight{for which type a
modification is available}. For instance any \lstinline!Car! might have a
\lstinline!Spoiler!, but only an \lstinline!OrdinaryCar! can be tuned with a
\lstinline!TurboCharger!. An \lstinline!abstract override! keyword combination is
required, because the methods we want to modify by \highlight{overriding} them
are \highlight{abstract}. This is the key to Scala's linearization model, which
is less dangerous than multiple inheritance. Since \highlight{super calls are
dynamically* bound}, a call to an abstract method might actually succeed, as
long as the modification trait is mixed in \emph{after} another trait or class that gives a concrete
definition to the method.
\newline
\newline
* Super calls are bound statically in classes. For instance you can always tell
at compile time exactly which method implementation will be invoked, when you write
\lstinline!super.toString!. In traits, however, they are dynamically bound.
\end{frame}

\begin{frame}[fragile]{Stackable Modifications - Modifying Functionality}
\begin{exampleblock}{Defining some cars}
\begin{lstlisting}
class Lamborghini(override val model: String) 
   extends SportsCar(model) with Spoiler
  
class BMW(override val model: String)
   extends OrdinaryCar(model) with Spoiler
   with EngineControlUnit with TurboCharger
\end{lstlisting}
\end{exampleblock}
\end{frame}

\begin{frame}[fragile]{Stackable Modifications - Modifying Functionality}
\begin{block}{The \lstinline!App! trait}
The \lstinline!App trait! is defined somewhere in the Scala library and is
implicitly imported in any .scala file along with lots of other useful
things. \lstinline!App! has a single member - the \lstinline!main! method.
If you mix \lstinline!App! into an \lstinline!object! the code from the primary
constructor (which is the code between the members of the object) is put into
the main method. So basically it defines the main method for you so you do not
have to do it yourself.
\end{block}
\begin{exampleblock}{Playing around with cars}
\begin{lstlisting}
object TestDrive extends App {
   // Lamborghini Murcielago 306 11000
   println(new Lamborghini("Murcielago"))
   // BMW M3-GTR 336 10000
   println(new BMW("M3-GTR"))
}
\end{lstlisting}
\end{exampleblock}
\end{frame}

\subsection{Intercepting Functionality}
\begin{frame}{Stackable Modifications - Intercepting Functionality}
The racing game showed how modifications can be made from the inside of the API.
This happened because the methods in question were not parametrized. This
section discusses how modifications can be made when they come from the outside
of the API, thus intercepting method arguments. We will use the Java collection
API to demonstrate Scala's seamless interoperability with Java.
\end{frame}

\begin{frame}[fragile]{Stackable Modifications - Intercepting Functionality}
\begin{exampleblock}{Modification trait}
\begin{lstlisting}
trait CaseIgnorance extends java.util.Set[String] {
  abstract override def add(e: String) =
    super.add(e.toLowerCase)

  abstract override def contains(e: Any) =
    super.contains(e.asInstanceOf[String].toLowerCase)

  abstract override def remove(e: Any) =
    super.remove(e.asInstanceOf[String].toLowerCase)
}
\end{lstlisting}
\end{exampleblock}
\end{frame}

\begin{frame}[fragile]{Stackable Modifications - Intercepting Functionality}
\begin{exampleblock}{Modification trait}
\begin{lstlisting}
trait Logging extends java.util.Set[String] {
  abstract override def add(e: String) = {
    println("Add: " + e)
    super.add(e)
  }

  abstract override def remove(e: Any) = {
    println("Remove: " + e)
    super.remove(e)
  }
}
\end{lstlisting}
\end{exampleblock}
\end{frame}

\begin{frame}[fragile]{Stackable Modifications - Intercepting Functionality}
\onslide<1->\lstinline!object Traitor extends App {!\\
\onslide<1->\lstinline!   import java.util._!\\
\lstinline!!\\
\onslide<2->\lstinline!   val firstLogThenIgnoreCase =!\\
\onslide<2->\lstinline!      new HashSet[String] with CaseIgnorance with Logging!\\
\onslide<4->\lstinline!   // prints out "Add: HI" then adds "hi" to the collection!\\
\onslide<3->\lstinline!   firstLogThenIgnoreCase.add("HI")!\\
\lstinline!!\\
\onslide<2->\lstinline!   val firstIgnoreCaseThenLog =!\\
\onslide<2->\lstinline!      new HashSet[String] with Logging with CaseIgnorance!\\
\onslide<4->\lstinline!   // adds "hi" to the collection then prints out "Add: hi"!\\
\onslide<3->\lstinline!   firstIgnoreCaseThenLog.add("HI")!\\
\lstinline!!\\
\onslide<5->\lstinline!   // prints out "true"!\\
\onslide<5->\lstinline!   println(firstLogThenIgnoreCase.contains("Hi") &&!\\
\onslide<5->\lstinline!           firstIgnoreCaseThenLog.contains("hI"))!\\
\onslide<1->\lstinline!}!
\end{frame}

\begin{frame}{Stackable Modifications - Intercepting Functionality}
Notice the significance of the order of the mixins. Given one modification trait you can
define two versions of sets (you can either mix it in or not). Given two modifications,
as in our example you can choose from one to five different variations of sets. The
linearization model gives you great flexibility.
\newline
\newline
As word of caution you need to remember that with great power comes great responsibility.
By controlling the order of mixins you control the state. Watching out for state
modifications has proven itself anything else than simple. For instance state
modifications of the sets are intentional, but a car with a spoiler and a turbo charger should not
be faster than a car with a turbo charger and a spoiler. It is not, so don't
panic.
\end{frame}

\section{Dependency Injection}
\begin{frame}[fragile]{Dependency Injection}
\begin{exampleblock}{Direct dependency}
\begin{lstlisting}
class B
class A(b: B)
\end{lstlisting}
\end{exampleblock}
\end{frame}

\begin{frame}[fragile]{Dependency Injection}
\begin{exampleblock}{Dependency Inversion}
\begin{lstlisting}
trait C
class B extends C
class A(b: C)
\end{lstlisting}
\end{exampleblock}
\end{frame}

\begin{frame}[fragile]{Dependency Injection}
\begin{block}{Dependency Injection}
The concrete instance of \lstinline!C! is chosen and injected by an external
dependency injection framework like Spring or google-guice.
\end{block}
\begin{exampleblock}{Dependency Injection}
\begin{lstlisting}
trait C
class B extends C
class A // magically depends upon C
\end{lstlisting}
\end{exampleblock}
\end{frame}

\begin{frame}[fragile]{Dependency Injection with traits}
\begin{block}{What is open recursion?}
Open recursion [\ldots] \emph{``is the ability for one method body to invoke
another method of the same object via a special variable called self or, in some
languages, this. The special behavior of self is that it is late-bound, allowing
a method defined in one class to invoke another method that is defined later, in
some subclass of the first.'' - Ralf Hinze}
\end{block}
\pause
\begin{exampleblock}{Syntax for changing the name of \lstinline!this!}
\lstinline!trait A {!\\
\lstinline!   self =>!\\
\lstinline!}!
\end{exampleblock}
\pause
\begin{exampleblock}{Syntax for changing the type of \lstinline!this!}
\lstinline!trait A {!\\
\lstinline!   self: SomeOtherType =>!\\
\lstinline!}!
\end{exampleblock}
\end{frame}

\begin{frame}[fragile]{Dependency Injection with traits}
\onslide<1->\begin{lstlisting}
trait Fruits {
   def fruitBasket = Set("apple", "orange", "banana")
}
\end{lstlisting}
\onslide<2->\lstinline!trait FruitSalad {!\\
\onslide<3->\lstinline!   self: Fruits => // now fruitBasket is in scope!\\
\lstinline!!\\  
\onslide<2->\lstinline!   def salad = fruitBasket.mkString( // fruitBasket is not in scope!\\
\onslide<2->\lstinline!   start = "Take one ",!\\
\onslide<2->\lstinline!   sep = " one ",!\\
\onslide<2->\lstinline!   end = " mix them all together and you are done.")!\\
\onslide<2->\lstinline!}!
\end{frame}

\begin{frame}[fragile]{Dependency Injection with traits}
\begin{alertblock}{Static type checking}
\begin{lstlisting}
scala> object Main extends App with FruitSalad
<console>:9: error: illegal inheritance;
 self-type Main.type does not conform to 
 FruitSalads selftype FruitSalad with Fruits
       object Main extends App with FruitSalad
                                    ^
\end{lstlisting}
\end{alertblock}
\begin{exampleblock}{Static type checking}
\begin{lstlisting}
object Main extends App with FruitSalad with Fruits {
   // Take one apple one orange one banana mix them all
   // together and you are done.
   println(salad)
}
\end{lstlisting}
\end{exampleblock}
\end{frame}

\begin{frame}[fragile]{Dependency Injection with traits}
\begin{exampleblock}{Circular Dependencies}
\begin{lstlisting}
trait Egg {
   self: Chicken =>
}

trait Chicken {
   self: Egg =>
}
\end{lstlisting}
\end{exampleblock}
\pause
\begin{exampleblock}{Circular Dependencies}
\begin{lstlisting}
class A extends Egg with Chicken
class B extends Chicken with Egg
\end{lstlisting}
\end{exampleblock}
\end{frame}

\begin{frame}[fragile]{Dependency Injection - The Cake Pattern}
\begin{exampleblock}{Database component}
\begin{lstlisting}
trait DatabaseComponent {
  def database: Database

  class Database {
    def addUser(user: String) = {
      println(user + " added to real db")
      true
    }
  }
}
\end{lstlisting}
\end{exampleblock}
\end{frame}

\begin{frame}[fragile]{Dependency Injection - The Cake Pattern}
\begin{exampleblock}{Service component}
\begin{lstlisting}
trait ServiceComponent {
  // ServiceComponent depends upon DatabaseComponent
  this: DatabaseComponent =>
  
  def service: Service
  
  class Service {
    def createUser(user: String) = {
      // database is provided via mixin
      database.addUser(user)
    }
  }
}
\end{lstlisting}
\end{exampleblock}
\end{frame}

\begin{frame}[fragile]{Dependency Injection - The Cake Pattern}
\begin{exampleblock}{Static wiring}
\begin{lstlisting}
trait ProductionEnvironment extends DatabaseComponent
  with ServiceComponent {
  def database = new Database
  def service = new Service
}

trait TestEnvironment extends DatabaseComponent
  with ServiceComponent {
  // Mock implementations on the next slide
  def database = new MockDatabase
  def service = new MockService
}
\end{lstlisting}
\end{exampleblock}
\end{frame}

\begin{frame}[fragile]{Dependency Injection - The Cake Pattern}
\begin{exampleblock}{Mocks}
\begin{lstlisting}
class MockDatabase extends Database {
  override def addUser(user: String) = {
    println(user + " added to mock db")
    true
  }
}

class MockService extends Service {
  override def createUser(user: String) = {
    println("service works")
    true
  }
}
\end{lstlisting}
\end{exampleblock}
\end{frame}

\begin{frame}[fragile]{Dependency Injection - The Cake Pattern}
\begin{exampleblock}{Tests}
\begin{lstlisting}
import org.scalatest._
import org.scalatest.matchers._

class ServiceTestSuitWithMockServiceAndMockDatabase
  extends TestEnvironment with FunSuite with ShouldMatchers {

  test("Test user creation") {
    // prints out "service works"
    service.createUser("TestUser") should be(true)
  }
}
\end{lstlisting}
\end{exampleblock}
\end{frame}

\begin{frame}[fragile]{Dependency Injection - The Cake Pattern}
\begin{exampleblock}{Tests}
\begin{lstlisting}
import org.scalatest._
import org.scalatest.matchers._

class ServiceTestSuitWithRealServiceAndMockDatabase
  extends TestEnvironment with FunSuite with ShouldMatchers {

  // real service but the Database is still a mock!
  override val service = new Service
  
  test("Test user creation") {
    // prints out "TestUser added to mock db"
    service.createUser("TestUser") should be(true)
  }
}
\end{lstlisting}
\end{exampleblock}
\end{frame}

\section{Summary}
\begin{frame}{Summary}
\begin{itemize}
  \item Traits are a very powerful concept
  \item Traits enable \highlight{component-driven} development
  \item Traits can have \highlight{concrete methods}
  \item Traits super calls are \highlight{dynamically bound}
  \item Scala does not support multiple inheritance
  \item Scala's \highlight{linearization} model provides great flexibility
  \item Mixin composition is a great way for adding behavior
  \item Traits give you dependency injection for free
\end{itemize}
\end{frame}