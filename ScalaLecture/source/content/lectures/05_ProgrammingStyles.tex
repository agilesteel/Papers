\part[Programming Styles]{Programming Styles}
\section{Paradigms VS Styles}
\begin{frame}{Paradigms VS Styles}
\begin{center}
Paradigms are about \alert{constraints}
\end{center}
\begin{center}
Styles are about \highlight{abilities}
\end{center}
\end{frame}

\begin{frame}{Programming Styles}
\begin{center}
OO \& FP are not only paradigms, they are also styles (and notations)
\end{center}
\begin{block}{OO buzzwords}
objects, classes, interfaces, methods, loops, statements, records, open
recursion, information hiding, packages, imports, modelling real world,
inheritance, polymorphism, ad hoc polymorphism, parametric polymorphism,
generics and so on\ldots
\end{block}
\pause
\begin{block}{FP buzzwords}
functions, pure functions, first-class functions, higher-order functions,
anonymous functions, lambdas, closures, (no) side effects, expressions,
referential transparency, parametric polymorphism, recursive path-dependent structural
types, algebraic types, product types, sum types, higher-ranked types,
multi-parameter typeclasses, monads, monoids, endomorphisms, catamorphisms and
so on\ldots
\end{block}
\end{frame}

\section{Declarative Programming}
\begin{frame}{Declarative Programming}
\begin{block}{What is Declarative Programming?}
\pause
Declarative Programming is a style of programming were you declare
\highlight{what} computation is going to be performed
\end{block}
\end{frame}

\begin{frame}{Mathematics}
\begin{block}{Why is it so complicated to understand how math works?}
Math is a declarative language. It expresses \highlight{what} has to be done,
but not \alert{how} it is done. Declaring what has to be done can be only
accomplished at a much \highlight{higher level of abstraction} than
instructing how something is actually done.
\end{block}
\pause
\begin{exampleblock}{Here is an example}
To be able to read this: \(f(x) = x^2\),\\ you need to be aware of this: \(x^2 =
x*x\)
\end{exampleblock}
\end{frame}

\begin{frame}[fragile]{SQL}
\begin{center}
SQL is a declarative language. SQL is easy to learn, but hard to master.
\end{center}
\pause
\begin{exampleblock}{Easy}
\begin{lstlisting}[language=sql]
select name from employees
\end{lstlisting}
\end{exampleblock}
\pause
\begin{alertblock}{Hard}
\begin{lstlisting}[language=sql]
select album.title from albums album
where row = (select id from rows row
              where row.id = album.row -- correlation
              and row.name = 'Asterix')
\end{lstlisting}
\end{alertblock}
\end{frame}

\begin{frame}[fragile]{Functional Programming}
\begin{center}
Functional Programming is a subset of Declarative Programming
\end{center}
\pause
\begin{exampleblock}{SQL example written in SQL}
\begin{lstlisting}[language=sql]
select name from employees
\end{lstlisting}
\end{exampleblock}
\pause
\begin{center}
Functional Programming is programming with functions
\end{center}
\pause
\begin{exampleblock}{SQL example written using a functional notation}
\begin{lstlisting}
employees.select(employee => employee.name)
\end{lstlisting}
\end{exampleblock}
\pause
\begin{exampleblock}{SQL example written using one of the Scala notations}
\begin{lstlisting}
employees map { _.name }
\end{lstlisting}
\end{exampleblock}
\end{frame}

\begin{frame}[fragile]{Iteration}
\begin{center}
Scala is a much more sophisticated language than Java. This means that
programming in Scala you can or maybe should use different concepts than in Java
to gain more benefit from the language.
\end{center}
\end{frame}

\begin{frame}[fragile]{Iteration}
\begin{exampleblock}{Fibonacci \emph{definition} in Math}
\[
  F_n = \left\{
  \begin{array}{l l}
    	0 & \quad \textrm{if $n$ = 0;}\\
    	1 & \quad \textrm{if $n$ = 1;}\\
    	F_n-1 + F_n-2 & \quad \textrm{if $n > 1$.}\\
  \end{array} \right.
\]
\end{exampleblock}
\pause
\begin{exampleblock}{Functional Fibonacci \emph{implementation} in Scala}
\begin{lstlisting}
def fibonacci(n: Int): Int =
   if (n == 0) 0
   else if (n == 1) 1
   else fibonacci(n - 1) + fibonacci(n - 2)
\end{lstlisting}
\end{exampleblock}
\end{frame}

\section{Imperative Programming}
\begin{frame}{Imperative Programming}
\begin{block}{What is Imperative Programming?}
\pause
Imperative Programming is a style of programming were you describe
\highlight{how} a computation is going to be performed
\end{block}
\end{frame}

\begin{frame}[fragile]{Iteration}
\begin{exampleblock}{Imperative Fibonacci \emph{implementation} in
Scala with a ``loop''}
\begin{lstlisting}
def fibonacci(n: Int) = {
   def loop = {
      var res = 0
      var f1 = 0
      var f2 = 1
      for (i <- 2 to n) { // 'for' is not just a loop in Scala
         res = f1 + f2
         f1 = f2
         f2 = res
      }
      res
   }

   if (n <= 1) n else loop
}
\end{lstlisting}
\end{exampleblock}
\end{frame}

\begin{frame}[fragile]{Iteration}
\begin{exampleblock}{Imperative recursive Fibonacci \emph{implementation} in
Scala}
\begin{lstlisting}
def fibonacci(n: Int) = {
   @scala.annotation.tailrec
   def recur(i: Int, f1: Int, f2: Int): Int =
      if (i == n) f2 else recur(i + 1, f2, f1 + f2)

   if (n <= 1) n else recur(1, 0, 1)
}
\end{lstlisting}
\end{exampleblock}
\pause
\begin{block}{Annotation}
The \lstinline!@scala.annotation.tailrec! is an optional annotation, which tells
the compiler to generate an \alert{error} if the definition contains a recursive call, which is
not in tail position
\end{block}
\end{frame}

\begin{frame}{Imperative VS Declarative}
\begin{center}
\highlight{Every} algorithms can be expressed either using the
\highlight{declarative} or the \highlight{imperative} style or a mixture of
both, but not every recursive algorithm has a tail-recursive version, because
not every algorithm can be expressed with a single loop.
\end{center}
\end{frame}

\begin{frame}{Imperative OO VS Declarative FP}
\begin{block}{Is imperative/declarative just another word for OO/FP?}
The \highlight{imperative style} is often confused or at least used in
connection with the \highlight{OO style}. The same goes for the
\highlight{declarative and FP} combination. Why is that so?
\end{block}
\pause
\begin{block}{Imperative OO}
Objects contain behaviour and \highlight{state}. Imperative style uses
\highlight{state}ments to handle \highlight{state}.
\end{block}
\pause
\begin{block}{Declarative FP}
Functions map inputs to outputs. Declarative style uses \highlight{expressions}
to produce values for the output of the functions.
\end{block}
\end{frame}

\begin{frame}{Imperative OO VS Declarative FP}
\begin{center}
Scala is a \highlight{blend} between the OO \& FP styles and thus a
\highlight{blend} between the imperative \& declarative programming styles
\end{center}
\begin{center}
If you try to \alert{stick to only one style} in Scala you will literally
\alert{fight the language (compiler)}
\end{center}
\begin{center}
Scala let's you \highlight{decide which style you prefer}, but encourages the
\highlight{functional} style
\end{center}
\end{frame}

\begin{frame}[fragile]{Imperative OO VS Declarative FP}
\begin{center}
We would like to port this program from Java to Scala
\end{center}
\begin{block}{This is Java}
\begin{lstlisting}{language=java}
public boolean isPrime(int n) {
   for (int i = 2; i * i <= n; i++)
      if (n % i == 0)
         return false;
   return true;
}
\end{lstlisting}
\end{block}
\end{frame}

\begin{frame}[fragile]{Imperative OO VS Declarative FP}
\begin{alertblock}{Compiles, but does not work as expected}
\begin{lstlisting}
def isPrime(n: Int) = {
   var i = 2
   while (i * i <= n) {
      if (n % i == 0)
         false
      i += 1
   }
   true
}

def isPrime(n: Int) = {
   for (i <- 2 to math.sqrt(n).toInt)
      if (n % i == 0)
         false
   true
}
\end{lstlisting}
\end{alertblock}
\end{frame}

\begin{frame}[fragile]{Imperative OO VS Declarative FP}
\begin{alertblock}{Does not compile}
\begin{lstlisting}
def isPrime(n: Int) = {
   var i = 2
   while (i * i <= n) {
      if (n % i == 0)
         return false
      i += 1
   }
   true
}

def isPrime(n: Int) = {
   for (i <- 2 to math.sqrt(n).toInt)
      if (n % i == 0)
         return false
   true
}
\end{lstlisting}
\end{alertblock}
\end{frame}

\begin{frame}[fragile]{Imperative OO VS Declarative FP}
\begin{block}{Compiles and works as expected, but does not look like Scala}
\begin{lstlisting}
def isPrime(n: Int): Boolean = {
   var i = 2
   while (i * i <= n) {
      if (n % i == 0)
         return false
      i += 1
   }
   true
}

def isPrime(n: Int): Boolean = {
   for (i <- 2 to math.sqrt(n).toInt)
      if (n % i == 0)
         return false
   true
}
\end{lstlisting}
\end{block}
\end{frame}

\begin{frame}[fragile]{Imperative OO VS Declarative FP}
\begin{exampleblock}{Compiles and works as expected}
\begin{lstlisting}
def isPrime(n: Int) =
   2 to n takeWhile { i => i * i <= n } forall { n % _ != 0 }

def isPrime(n: Int) =
   2 to math.sqrt(n).toInt forall { n % _ != 0 }
\end{lstlisting}
\end{exampleblock}
\end{frame}

\section{Syntax}
\subsection{Hierarchies}
\begin{frame}[fragile]{Hierarchies}
\onslide<1->
\begin{lstlisting}
abstract class Animal
class Dog extends Animal
class Cat extends Animal
\end{lstlisting}

\onslide<2->
\begin{lstlisting}
scala> val dog = new Dog
dog: Dog = Dog@448d9973

scala> val cat = new Cat
cat: Cat = Cat@5489f73f
\end{lstlisting}

\onslide<3->
\begin{lstlisting}
scala> val animal: Animal = dog
animal: Animal = Dog@48d9973

scala> val animal: Animal = cat
animal: Animal = Cat@5489f73f
\end{lstlisting}
\end{frame}

\begin{frame}[fragile]{Hierarchies - abstract declarations}
\onslide<1->
\begin{lstlisting}
abstract class Animal {
   def name: String // avoid abstract vals, use defs instead
   var size: Int
   def eat(food: String): Unit
}
\end{lstlisting}

\onslide<2->
\begin{lstlisting}
class Dog(val name: String, var size: Int) extends Animal {
   def eat(food: String) = println("Eating " + food)
}

class Cat(var size: Int) extends Animal {
   val name = "Garfield"
   def eat(food: String) = {
      def response = if (food.toLowerCase == "lasagna")
         "Eating " + food else "I'm not eating that garbage!"     
      println(response)
   }
}
\end{lstlisting}
\end{frame}

\begin{frame}[fragile]{Hierarchies - concrete definitions}
\onslide<1->
\begin{lstlisting}
abstract class Animal {
   def name = "Default name"
   var size = 1
   def eat(food: String) = println("Eating " + food)
}
\end{lstlisting}

\onslide<2->
\begin{lstlisting}
class Dog(override val name: String) extends Animal
\end{lstlisting}

\onslide<3->
\begin{lstlisting}
class Cat(override val name: String, s: Int) extends Animal {
   size = s
   override def eat(food: String) = {
      def response = if (food.toLowerCase == "lasagna")
         "Eating " + food else "I'm not eating that garbage!"     
      println(response)
   }
}
\end{lstlisting}
\end{frame}

\begin{frame}[fragile]{Hierarchies - be aware of var gotchas}
\begin{lstlisting}
abstract class Animal {
   def name = "Default name"
   var size = 1
}

class Dog(override val name: String,
           override var size: Int) extends Animal 
           // overriding a var
\end{lstlisting}
\begin{alertblock}{Compiler error}
\begin{lstlisting}
overriding variable size in class Animal of type Int;
variable size cannot override a mutable variable
override var size: Int) extends Animal
              ^
\end{lstlisting}
\end{alertblock}
\end{frame}

\begin{frame}[fragile]{Hierarchies - be aware of var gotchas}
\begin{lstlisting}
abstract class Animal {
   def name = "Default name"
   var size = 1
}

class Dog(override val name: String,
           var size: Int) extends Animal 
           // name conflict
\end{lstlisting}
\begin{alertblock}{Compiler error}
\begin{lstlisting}
overriding variable size in class Animal of type Int;
variable size needs override modifier
var size: Int) extends Animal
    ^
\end{lstlisting}
\end{alertblock}
\end{frame}

\begin{frame}[fragile]{Hierarchies - constructors + default arguments}
\onslide<1->
\begin{lstlisting}
abstract class Animal(var size: Int = 1) {
   def name = "Default name"
}
\end{lstlisting}

\onslide<2->
\begin{lstlisting}
// ignoring default argument
class Dog(override val name: String, s: Int) extends Animal(s)
\end{lstlisting}

\onslide<3->
\begin{lstlisting}
// using default argument
class Cat(override val name: String) extends Animal
\end{lstlisting}

\onslide<4->
\begin{lstlisting}
// ignoring default argument + using named parameter
class Frog(override val name: String,
                              s: Int) extends Animal(size = s)
\end{lstlisting}
\end{frame}

\subsection{Singletons}
\begin{frame}[fragile]{Singletons}
\begin{lstlisting}
class Fruit

object Apple extends Fruit {
   override def toString = "Apple"
}

object Orange extends Fruit {
   override def toString = "Orange"
}
\end{lstlisting}
\end{frame}

\begin{frame}[fragile]{Singletons}
\begin{lstlisting}
object FirstApplication {
   def main(args: Array[String]) = {
      println(Orange) // Orange
      val apple = Apple
      println(apple == Apple) // true (object equality)
      println(apple eq Apple) // true (referential equality)
      eatOrange(Orange) // Eating Orange
      eatFruit(apple) // Eating Apple
   }
   
   def eatOrange(orange: Orange.type) = { // rarely used
      println("Eating " + orange)
   }
   
   def eatFruit(fruit: Fruit) = {
      println("Eating " + fruit)
   }
}
\end{lstlisting}
\end{frame}

\begin{frame}[fragile]{Singletons - behind the scenes}
\begin{lstlisting}
object Orange

// This is not what actually happens,
// but it helps to understand the concept

val Orange = {
   class SynthesizedOrange
   new SynthesizedOrange
}

// These are also only roughly equivalent

Orange.type == SynthesizedOrange
\end{lstlisting}

\end{frame}

\begin{frame}[fragile]{Singletons - companion objects are new statics}
\begin{block}{What is a companion object?}
A companion object is an object, which has the \highlight{same name} as it's
companion class. Companion objects \alert{must} be contained in the
\highlight{same file} as their respective classes.
\end{block}
\pause
\begin{lstlisting}
object FootballPlayer {
   var positions = List[Int]()
   def apply(position: Int) = new FootballPlayer(position)
}

class FootballPlayer(val position: Int) {
   import FootballPlayer._
   require(position >= 1 && position <= 11)
   require(positions forall { _ != position })
   positions = position :: positions
}
\end{lstlisting}
\end{frame}

\begin{frame}[fragile]{Singletons - companion objects are new statics}
\begin{lstlisting}
object SinglePlayerPerPosition {
   def main(args: Array[String]) = {
      for (i <- 1 to 11)
         FootballPlayer(i)

      FootballPlayer(7) // throws an exception
   }
}
\end{lstlisting}
\end{frame}

\subsection{Packages, Imports, Visibility}

\begin{frame}[fragile]{Packages}
\begin{block}{Packages are namespaces}
Packages in Scala are more like namespaces in C\#. This means they can be
nested. Scala packages can look either like Java packages or C\# namespaces or
both.
\end{block}
\pause
\begin{lstlisting}
package de.htwgkonstanz {
   package lecture {
      class Scala
   }
   
   package presentation {
      class ScalaPresentation {
         val subject = new lecture.Scala
      }
   }
}
\end{lstlisting}
\end{frame}

\begin{frame}[fragile]{Imports}
\begin{block}{In Scala, imports:}
\begin{itemize}
  \item may appear \highlight{anywhere}
  \item may refer to \highlight{objects} (singleton or regular) in addition to
  packages
  \item let you \highlight{rename} or \highlight{hide} some of the imported
  members
\end{itemize}
\end{block}
\pause
\begin{lstlisting}
object Fruits {
   object Apple
   object Orange
   object Pear
}

// Somewhere else in the code base
import Fruits.{Pear => _, _} // imports everything, but Pear
\end{lstlisting}
\end{frame}

\begin{frame}[fragile]{Imports}
\begin{lstlisting}
import java.sql.{Date => SDate} // renames sql.Date to SDate

class Dog(val name: String)

class ImportExample{
   def printName(dog: Dog) = {
      import dog._ // imports members of dog
      println("This is " + name);
   }
   
   def printDate(date: SDate) = {
      println(date)
   }
}
\end{lstlisting}
\end{frame}

\begin{frame}[fragile]{Visibility}
\begin{lstlisting}
package bobsrockets {
   package navigation {
      private[bobsrockets] class Navigator {
         private[navigation] def useStarChart() = {}
         class LegOfJourney {
            private[Navigator] val distance = 100
         }
         private[this] var speed = 200
      }
   }
   package launch {
      import navigation._
      object Vehicle {
         private[launch] val guide = new Navigator
      }
   }
}
\end{lstlisting}
\end{frame}

\begin{frame}[fragile]{Visibility}
\begin{block}{Effects of private qualifiers on \lstinline!LegOfJourney.distance!}
\begin{tabular}{l|l}
no access modifier & public access \\
\lstinline!private[bobsrockets]! & access within outer package \\
\lstinline!private[navigation]! & same as package visibility in Java \\
\lstinline!private[Navigator]! & same as private in Java \\
\lstinline!private[LegOfJourney]! & same as private in Scala \\
\lstinline!private[this]! & access only from same object \\
\end{tabular}
\end{block}
\end{frame}

\section{Summary}
\begin{frame}[fragile]{Summary}
\begin{itemize}
  \item Styles are \highlight{features}
  \item Features are about \highlight{abilities}
  \item Declarative programming is about \highlight{what} has to be done
  \item Imperative programming is about \highlight{how} something has to be done
  \item scalac can optimize \highlight{tail-recursion}
  \item \highlight{Every} declarative algorithm can be described imperatively
  \item \highlight{Every} imperative algorithm can be described declaratively
\end{itemize}
\end{frame}

\begin{frame}[fragile]{Summary}
\begin{itemize}
  \item \alert{Avoid} \lstinline!abstract val!, use \lstinline!def! instead
  \item A \lstinline!var! cannot be overridden
  \item Scala's methods/classes have \highlight{named} parameters
  \item Scala's parameters can have \highlight{default} arguments
  \item \lstinline!==! checks for \highlight{object equality}
  \item \lstinline!eq! checks for \highlight{referential equality}
  \item \lstinline!ne! checks for \highlight{referential inequality}
  \item \highlight{packages} are namespaces
  \item \highlight{imports} are very flexible
  \item \highlight{visibility} is very granular
\end{itemize}
\end{frame}