\part[Programming Styles]{Programming Styles}
\section{Paradigms VS Styles}
\begin{frame}{Paradigms VS Styles}
\begin{center}
Paradigms are about \alert{constraints}
\end{center}
\begin{center}
Styles are about \highlight{abilities}
\end{center}
\end{frame}

\begin{frame}{Programming Styles}
\begin{center}
OO \& FP are not only paradigms, they are also styles (and notations)
\end{center}
\begin{block}{OO buzzwords}
objects, classes, interfaces, methods, loops, statements, records, open
recursion, information hiding, packages, imports, inheritance, polymorphism, ad
hoc polymorphism, parametric polymorphism, generics and so on\ldots
\end{block}
\begin{block}{FP buzzwords}
functions, pure functions, first-class functions, higher-order functions,
anonymous functions, lambdas, closures, (no) side effects, expressions,
referential transparency, parametric polymorphism, recursive path-dependent structural
types, algebraic types, product types, sum types, higher-ranked types,
multi-parameter typeclasses, monads, monoids, endomorphisms, catamorphisms and
so on\ldots
\end{block}
\end{frame}

\section{Declarative Programming}
\begin{frame}{Declarative Programming}
\begin{block}{What is Declarative Programming?}
Declarative Programming is a style of programming were you declare
\highlight{what} computation is going to be performed
\end{block}
\end{frame}

\begin{frame}{Mathematics}
\begin{block}{Why is it so complicated to understand how math works?}
Math is a declarative language. It expresses \highlight{what} has to be done,
but not \alert{how} it is done. Declaring what has to be done can be only
accomplished at a much \highlight{higher level of abstraction} than
instructing how something is actually done.
\end{block}
\pause
\begin{exampleblock}{Here is an example}
To be able to read this: \(f(x) = x^2\),\\ you need to be aware of this: \(x^2 =
x*x\)
\end{exampleblock}
\end{frame}

\begin{frame}[fragile]{SQL}
\begin{center}
SQL is a declarative language. SQL is easy to learn, but hard to master.
\end{center}
\pause
\begin{exampleblock}{Easy}
\begin{lstlisting}[language=sql]
select name from employees
\end{lstlisting}
\end{exampleblock}
\pause
\begin{alertblock}{Hard}
\begin{lstlisting}[language=sql]
select album.title from albums album
where row = (select id from rows row
              where row.id = album.row -- correlation
              and row.name = 'Asterix')
\end{lstlisting}
\end{alertblock}
\end{frame}

\begin{frame}[fragile]{Functional Programming}
\begin{center}
Functional Programming is a subset of Declarative Programming.
\end{center}
\pause
\begin{exampleblock}{SQL example written in SQL}
\begin{lstlisting}[language=sql]
select name from employees
\end{lstlisting}
\end{exampleblock}
\pause
\begin{exampleblock}{SQL example written using a functional notation}
\begin{lstlisting}
employees.select(employee => employee.name)
\end{lstlisting}
\end{exampleblock}
\end{frame}

\begin{frame}[fragile]{Iteration}
\begin{center}
Scala is a much more sophisticated language than Java. This means that
programming in Scala you can or maybe should use different concepts than in Java
to gain more benefit from the language.
\end{center}
\end{frame}

\begin{frame}[fragile]{Iteration}
\begin{exampleblock}{Fibonacci \emph{definition} in Math}
\[
  F_n = \left\{
  \begin{array}{l l}
    	0 & \quad \textrm{if $n$ = 0;}\\
    	1 & \quad \textrm{if $n$ = 1;}\\
    	F_n-1 + F_n-2 & \quad \textrm{if $n > 1$.}\\
  \end{array} \right.
\]
\end{exampleblock}
\pause
\begin{exampleblock}{Fibonacci \emph{implementation} in Scala}
\begin{lstlisting}
def fibonacci(n: Int): Int =
   if (n == 0) 0
   else if (n == 1) 1
   else fibonacci(n - 1) + fibonacci(n - 2)
\end{lstlisting}
\end{exampleblock}
\end{frame}

\section{Imperative Programming}
\begin{frame}{Imperative Programming}
\begin{block}{What is Imperative Programming?}
Imperative Programming is a style of programming were you describe
\highlight{how} a computation is going to be performed
\end{block}
\end{frame}
\section{Summary}
\begin{frame}{Summary}
\begin{itemize}
  \item 
\end{itemize}
\end{frame}